% Name, phone, email, address
{\bf\huge Dongkwan Kim} \vspace{1em}\\
\noindent\begin{tabular}[t]{@{}l}
  Postdoctoral Fellow, SSLab\\
  School of Cybersecurity and Privacy \\
  Georgia Institute of Technology
\end{tabular}
\hfill
\begin{tabular}[t]{r@{}}
\\
Email: \email{0xdkay@gmail.com} \\
Homepage: \url{https://0xdkay.me} \\
\end{tabular}

% ========================================
% Summary
% ========================================
\sectiontitle{Summary}

I am a security engineer and researcher who uncovers \textbf{cross-domain threats} in AI-integrated systems,
IoT, cellular networks, and cyber-physical infrastructure.

\begin{itemize}[topsep=-1pt,itemsep=0pt,parsep=0pt,partopsep=0pt,leftmargin=1em]

\item 
\textbf{DARPA AIxCC finalist:} Designed and implemented an \emph{agentic LLM framework} for automatic bug discovery and exploit generation across oss-fuzz codebases.

\item
\textbf{Samsung SDS (special contract):}
Hardened security across various product types and service architectures including AI-powered features
(e.g., securing prompt-injection chains that escalate to remote-code execution, impersonation, and sensitive-data leakage).

\item
\textbf{Research \& IP:}
9 top-tier papers (USENIX Security, CCS, NDSS, TSE, TMC),
7 patents, and 17 industry / government security projects delivered.

\item
\textbf{CTF leadership:}
Led KAIST graduate hacking team \emph{KaisHack},
organized Samsung CTF 2017/18, reached the DEF CON CTF finals five times, and won multiple CTFs.

\end{itemize}

My mission is to translate cutting-edge research into \textit{real-world defences} that secure large-scale systems.


% I am a passionate, self-motivated security researcher.
% %at the School of Electrical Engineering, KAIST.
% My goal is to secure the Internet of Things (IoT) ecosystem.
% %To achieve this, I am actively studying on
% %systematizing the vulnerability analysis process of IoT devices.
% %
% To achieve this, I have experienced
% (1) smart home systems (\eg, wired/wireless routers, IP cameras, smart TVs),
% (2) mobile systems (\eg, Android apps, baseband software, wearables),
% (3) cellular infrastructures (\eg, specifications, charging policy, VoLTE, femtocells),
% (4) smart infrastructures (\eg, automobiles, drones), and
% (5) blockchain systems (\eg, Bitcoin, Ethereum, EOS).
% %
% To broaden my horizons,
% I competed in various hacking contests:
% (1) U.S. (\eg, DEFCON, Plaid CTF),
% (2) South Korea (\eg, Codegate, Whitehat Contest, HDCON), and
% (3) China (\eg, 0CTF).
% %
% Additionally, I have co-operated with the KAIST CERT team
% on investigating intrusion cases for over ten years.

% ========================================
% Research Interest
% ========================================
% \sectiontitle{Research Interest}
% I am interested in systematizing and automating vulnerability discovery of IoT devices
% using ML-assisted strategies, especially in:

% \begin{timeitemize}{Fundamental binary analysis}{}
%     \item Discover bugs with binary code similarity
%     \item Applying NLP to assembly languages
% \end{timeitemize}
% \begin{timeitemize}{Cellular network as a target}{}
%     \item Analyzing security violations in specifications and actual implementations
%     \item Discovering inconsistencies between specifications and actual implementations
%     \item Investigating privacy-leaking side channels
% \end{timeitemize}
% \begin{timeitemize}{Cyber-physical system as a target}{}
%     \item Discovering security (or safety) violations under adversarial environments
%     \item Building an end-to-end fuzzer for sensing and actuation logic
% \end{timeitemize}

% \clearpage
% ========================================
% Work Experience
% ========================================
\sectiontitle{Work Experience}
\begin{timeitemize}[Postdoctoral Fellow, Atlanta, GA]{Georgia Tech}{Feb. 2025 -- Present}
    \item As a finalist of DARPA AIxCC, I led design and implementation of LLM-powered exploit generation and fuzzing agents, significantly improving team productivity and analysis throughput through robust automation and tooling.
    \item Manager: Prof. Taesoo Kim
\end{timeitemize}

\begin{timeitemize}[Senior Engineer, South Korea]{Samsung Security Center, Samsung SDS}{Aug. 2022 -- Dec. 2024}
    \item Drove Red Team efforts to proactively identify and mitigate security threats across all Samsung affiliates' products and services, including AI-integrated systems, IoT/embedded devices, Android applications, and kernel-level mitigations.
    Expanded the traditional Red Team perspective to AI safety issues and implemented an automated AI-safety checker leveraging safety guardrail frameworks.
    % \item Manager: Classified
\end{timeitemize}

\begin{timeitemize}[Postdoctoral Researcher, South Korea]{KAIST}{Mar. 2022 -- Jul. 2022}
    \item Conducted advanced research on smartphone baseband authentication bypass, acoustic and EMI signal injection attacks against drone sensors, and recovery techniques for spoofed signals.
    \item Manager: Prof. Yongdae Kim
\end{timeitemize}

\begin{timeitemize}[Research Intern, South Korea]{Pinion Industries}{Dec. 2013 -- Feb. 2014}
    \item An automotive software and security startup. I analyzed and exploited in-vehicle components (network systems, AVN, telematics, smart keys, ECUs), including achieving remote code execution on AVN systems and investigating smart key cloning for potential theft scenarios.
    % \item Manager: Kichang Yang (Head of Hyundai Motors Group Security Center as of Mar. 2025)
    % \item CEO: Woongjun Jang (VP at Hyundai Motor Company as of Jan. 2021)
\end{timeitemize}

\begin{timeitemize}[Student Senior, South Korea]{KAIST CERT}{Sep. 2010 -- Aug. 2012}
    % \item Periodic pen-testing on servers under the KAIST domain ($\ast$.kaist.ac.kr)
    \item Investigated and analyzed security incidents, including identifying and attributing an attack that led to the apprehension of the perpetrator by law enforcement.
\end{timeitemize}

\clearpage
% ========================================
% Education
% ========================================
\sectiontitle{Education}
\begin{timeitemize}[South Korea]{Korea Advanced Institute of Science and Technology (KAIST)}{}
    \timeitem{Ph.D. in School of Electrical Engineering}{Mar. 2016 -- Feb. 2022}
    \subitem{Thesis Title:
        Improving Large-Scale Vulnerability Analysis of IoT Devices with
    Heuristics and Binary Code Similarity}
    \subitem{Advisor: Prof. Yongdae Kim}

    \timeitem{M.S. in School of Electrical Engineering}{Mar. 2014 -- Feb. 2016}
    \subitem{Thesis Title: Dissecting VoLTE: Exploiting Free Data Channels and Security Problems}
    \subitem{Advisor: Prof. Yongdae Kim}

    \timeitem{B.S. in School of Computing}{Feb. 2010 -- Feb. 2014}
\end{timeitemize}

\begin{timeitemize}[France]{EURECOM}{}
    \timeitem{Visiting Scholar in Software and System Security}{Jun. 2014 -- Jul. 2014}
    \subitem{Learned embedded device analysis techniques, particularly for
    debugging interfaces}
    \subitem{Advisor: Prof. Aur\'elien Francillon}
\end{timeitemize}


% ========================================
% Honors & Awards
% ========================================
\sectiontitle{Honors \& Awards}
\begin{timeitemize}{Hacking Contests (\ie, Capture-the-flag, CTF)}{}
    \timeitem{Finalist, DEFCON 27 CTF}{(Team KaisHack GoN) Aug. 2019}
    \timeitem{Finalist, DEFCON 26 CTF}{(Team KaisHack+PLUS+GoN) Aug. 2018}
    \timeitem{1st place (\$20,000), HDCON CTF}{(Team maxlen) Nov. 2017}
    \timeitem{1st place (\$30,000), Whitehat Contest}{(Team Old GoatskiN) Nov. 2017}
    \timeitem{3rd place (\$5,000), Codegate CTF}{(Team Old GoatskiN) Apr. 2017}
    \timeitem{Finalist, DEFCON 24 CTF}{(Team KaisHack GoN) Aug. 2016}
    \timeitem{1st place (\$20,000), Whitehat Contest}{(Team SysSec) Nov. 2014}
    \timeitem{Finalist, DEFCON 22 CTF}{(Team KAIST GoN) Aug. 2014}
    \timeitem{Silver prize (\$2,000), HDCON CTF}{(Team GoN) Dec. 2013}
    \timeitem{1st place (\$20,000), Whitehat Contest}{(Team KAIST GoN) Oct. 2013}
    \timeitem{Finalist, DEFCON 20 CTF}{(Team KAIST GoN) Jul. 2012}
    \timeitem{Silver prize (\$2,000), HDCON CTF}{(Team KAIST GoN) Jul. 2012}
    \timeitem{3rd place (\$5,000), Codegate CTF 2012}{(Team KAIST GoN) Apr. 2012}
    \timeitem{1st place (\$10,000), ISEC CTF}{(Team GoN) Sep. 2011}
    \timeitem{1st place (\$1,000), PADOCON CTF}{(Team GoN) Jan. 2011}
\end{timeitemize}

\begin{timeitemize}{Academic Awards}{}
    \timeitem{Best Paper Award, CISC-W}{Nov. 2020}
    \subitem{Title: Standard-based User Identifier Mapping Attack Prevention Method for LTE Network}

    \timeitem{Best Presentation Award, A3 Security Workshop}{Feb. 2016}
    \subitem{Title: Breaking and Fixing VoLTE: Exploiting Hidden Data Channels and Mis-implementations}

    \timeitem{Best Paper Award, WISA}{Aug. 2015}
    \subitem{Title: BurnFit: Analyzing and Exploiting Wearable Devices}
\end{timeitemize}

\begin{timeitemize}{Reported Security Vulnerabilities}{}
    \timeitem{CVE-2015-6614, Android telephony privilege escalation, Google}{Oct. 2015}
\end{timeitemize}

\begin{timeitemize}{Certificates}{}
    \timeitem{Engineer Information Security (\ie, 정보보안기사), South Korea}{Jun. 2016}
    \timeitem{Engineer Information Processing (\ie, 정보처리기사), South Korea}{May 2013}
\end{timeitemize}

\clearpage
\begin{timeitemize}{Scholarships}{}
    \timeitem{National Scholarship (Science and Engineering), Korea Student Aid Foundation}{Feb. 2010 -- Feb. 2020}
\end{timeitemize}

% \clearpage
% ========================================
% Patents
% ========================================
\sectiontitle{Patents}
\begin{timeitemize}{International Registrations}{}
    \timeitem{US 10111120}{Oct. 2018}
    \subitem{Method and Apparatus for Checking Problem in Mobile Communication Network}
\end{timeitemize}

\begin{timeitemize}[South Korea]{Domestic Registrations}{}
    \timeitem{KR 10-2514809}{Mar. 2023}
    \subitem{VIDEO IDENTIFICATION METHOD IN LTE NETWORKS AND THE SYSTEM THEREOF}

    \timeitem{KR 10-2418212}{Jul. 2022}
    \subitem{ARCHITECTURE-INDEPENDENT SIMILARITY MEASURING METHOD FOR PROGRAM FUNCTION}
    
    \timeitem{KR 10-2415494}{Jun. 2022}
    \subitem{Emulation based security analysis method for embedded devices}

    \timeitem{KR 10-2333866}{Nov. 2021}
    \subitem{Method and Apparatus for Checking Problem in Mobile Communication Network}

    \timeitem{KR 10-1972825}{Apr. 2019}
    \subitem{Method and apparatus for automatically analyzing vulnerable point
        of embedded appliance by using hybrid analysis technology, and computer
        program for executing the method}

    \timeitem{KR 10-1868836}{Jun. 2018}
    \subitem{A method to attack commercial drones using the resonance effect of
    gyroscopes by sound waves}
\end{timeitemize}

\begin{timeitemize}{Applications}{}
    \timeitem{KR 10-2022-0132964}{Oct. 2022}
    \subitem{ANTI-DRONE SYSTEM THROUGH COMMUNICATION DISTORTION BETWEEN SENSOR AND CONTROL UNIT AND ITS OPERATION METHOD}
    
    \timeitem{KR 10-2021-0168382}{Nov. 2021}
    \subitem{Method and System for Automatically Analyzing Bugs in Cellular Baseband Software using Comparative Analysis based on Cellular Specifications}
    
    \timeitem{KR 10-2021-0136352}{Oct. 2021}
    \subitem{METHOD FOR PREVENTING MAPPING OF USER IDENTIFIERS IN MOBILE COMMUNICATION SYSTEM AND THE SYSTEM THEREOF}
    
    \timeitem{KR 10-2021-0040795}{Mar. 2021}
    \subitem{ANALYSIS SYSTEM FOR DETECTION OF SIP IN VoLTE AND THE METHOD THEREOF}

    \timeitem{KR 10-2020-0177062}{Dec. 2020}
    \subitem{Analysis method for detection of SIP implementation vulnerability in VoLTE}

    \timeitem{KR 10-2020-0133926}{Oct. 2020}
    \subitem{Method to prevent mapping of user identifiers in mobile communication system}

    \timeitem{KR 10-2020-0133925}{Oct. 2020}
    \subitem{APPARATUS AND METHOD FOR VIDEO TITLE IDENTIFICATION OF MOBILE
    COMMUNICATION NETWORK USING ENCRYPTED TRAFFIC MONITORING}

    \timeitem{KR 10-2019-0005131}{Jan. 2019}
    \subitem{Large-scale honeypot system IoT botnet analysis}

    \timeitem{KR 10-2018-0036403}{Mar. 2018}
    \subitem{Dynamic analysis method for malicious embedded firmware detection}

    \timeitem{KR 10-2018-0036055}{Mar. 2018}
    \subitem{Emulation based security analysis method for embedded devices}

    \timeitem{KR 10-2018-0037291}{Mar. 2018}
    \subitem{Binary-Level Virtual Function Call Protection Method by Saving Type Information}

    \timeitem{KR 10-2018-0034616}{Mar. 2018}
    \subitem{ARCHITECTURE-INDEPENDENT SIMILARITY MEASURING METHOD FOR PROGRAM FUNCTION}
\end{timeitemize}

% \clearpage
% ========================================
% Publications
% ========================================
\sectiontitle{Publications (International)}
($\ast$: co-first authors)

9 papers in top-tier conferences and journals (USENIX Security, CCS, NDSS, TSE, TMC)

\begin{enumerate}[leftmargin=1.5em]
    \item \textbf{BaseComp: A Comparative Analysis for Integrity Protection
    in Cellular Baseband Software} \\
        {\small
            Eunsoo Kim*, Min Woo Baek*, CheolJun Park, \iam{Dongkwan Kim}, Yongdae Kim, and Insu Yun \\
            Proceedings of the 32nd USENIX Security Symposium (Security'23) \\
            Acceptance rate: 29.22\% (422 of 1,444)
        }
    \hfill {\small Aug. 2023}
    
    \item \textbf{Un-Rocking Drones: Foundations of Acoustic
Injection Attacks and Recovery Thereof} \\
        {\small
            Jinseob Jung, \iam{Dongkwan Kim}, Joonha Jang, Juhwan Noh, Changhun Song, and Yongdae Kim \\
            Proceedings of the 2023 Annual Network and Distributed System Security Symposium (NDSS'23) \\  
            Acceptance rate: 16.18\% (94 of 581)
        }
    \hfill {\small Mar. 2023}

    \item \textbf{Paralyzing Drones via EMI Signal Injection on
Sensory Communication Channels} \\
        {\small
            Junha Jang, ManGi Cho, Jaehoon Kim, \iam{Dongkwan Kim}, and Yongdae Kim \\
            Proceedings of the 2023 Annual Network and Distributed System Security Symposium (NDSS'23) \\ 
            Acceptance rate: 16.18\% (94 of 581)
        }
    \hfill {\small Mar. 2023}

    \item \textbf{Watching the Watchers: Practical Video Identification Attack
    in LTE Networks} \\
        {\small
            Sangwook Bae, Mincheol Son, \iam{Dongkwan Kim}, CheolJun Park, Jiho Lee, Sooel Son, and Yongdae Kim \\
            Proceedings of the 31st USENIX Security Symposium (Security'22) \\
            Acceptance rate: 18.10\% (256 of 1,414)
        }
    \hfill {\small Aug. 2022}

    \item \textbf{Revisiting Binary Code Similarity Analysis using Interpretable Feature Engineering and Lessons Learned} \\
        {\small
            \iam{Dongkwan Kim}, Eunsoo Kim, Sang Kil Cha, Sooel Son, and Yongdae Kim \\
            IEEE Transactions on Software Engineering
            (TSE'22)
        }
    \hfill {\small Jul. 2022}

    \item \textbf{Improving Large-Scale Vulnerability Analysis of IoT Devices with
    Heuristics and Binary Code Similarity} \\
        {\small
            \iam{Dongkwan Kim} \\
            Ph.D. Thesis, KAIST
        }
    \hfill {\small Daejeon, South Korea, Feb. 2022}


    \item \textbf{Enabling the Large-Scale Emulation of Internet of Things
        Firmware With Heuristic Workarounds} \\
        {\small
            \iam{Dongkwan Kim}, Eunsoo Kim, Mingeun Kim, Yeongjin Jang, and Yongdae Kim \\
            IEEE Security \& Privacy
        }
        \hfill {\small May 2021}

    % \clearpage
    \item \textbf{BaseSpec: Comparative Analysis of Baseband Software and Cellular Specifications for L3 Protocols} \\
        {\small
            \iam{Dongkwan Kim$^\ast$}, Eunsoo Kim$^\ast$, CheolJun Park, Insu Yun, and Yongdae Kim \\
            Proceedings of the 2021 Annual Network and Distributed System Security Symposium (NDSS'21) \\
            Acceptance rate: 15.18\% (87 of 573)
        }
        \hfill {\small Virtual, Feb. 2021}

    % \clearpage
    \item \textbf{FirmAE: Towards Large-Scale Emulation of IoT Firmware for Dynamic Analysis} \\
        {\small
            Mingeun Kim, \iam{Dongkwan Kim}, Eunsoo Kim, Suryeon Kim, Yeongjin Jang, and Yongdae Kim \\
            Proceedings of the 2020 Annual Computer Security Applications Conference (ACSAC'20)\\
            Acceptance rate: 23.18\% (70 of 302)
        }
        \hfill {\small Virtual, Dec. 2020}

    \item \textbf{Who Spent My EOS? On the (In)Security of Resource Management of EOS.IO} \\
        {\small
            Sangsup Lee, Daejun Kim, \iam{Dongkwan Kim}, Sooel Son, and Yongdae Kim \\
            Proceedings of the 13th USENIX Workshop on Offensive Technologies \\ (WOOT'19)
        } 
        \hfill {\small Santa Clara, CA, Aug. 2019}

    \item \textbf{Peeking over the Cellular Walled Gardens - A Method for Closed Network Diagnosis} \\
        {\small
            Byeongdo Hong, Shinjo Park, Hongil Kim, \iam{Dongkwan Kim}, Hyunwook Hong, Hyunwoo Choi, Jean-Pierre Seifert, Sung-Ju Lee, and Yongdae Kim \\
            IEEE Transactions on Mobile Computing (TMC'18)
        }
        \hfill {\small Feb. 2018}

    \item \textbf{When Cellular Networks Met IPv6: Security Problems of Middleboxes in IPv6 Cellular Networks} \\
        {\small
            Hyunwook Hong, Hyunwoo Choi, \iam{Dongkwan Kim}, Hongil Kim, Byeongdo Hong, Jiseong Noh, and Yongdae Kim \\
            Proceedings of the 2nd IEEE European Symposium on Security and Privacy (EuroS\&P'17)\\
            Acceptance rate: 19.58\% (38 of 194)
        }
        \hfill {\small Paris, France, Apr. 2017}

    %\clearpage
    \item \textbf{Pay As You Want: Bypassing Charging System in Operational Cellular Networks} \\
        {\small
            Hyunwook Hong, Hongil Kim, Byeongdo Hong, \iam{Dongkwan Kim}, Hyunwoo Choi, Eunkyu Lee, and Yongdae Kim \\
            Proceedings of the 17th International Workshop on Information Security Applications \\ (WISA'16)
        }
        \hfill {\small Jeju, South Korea, Aug. 2016}

    \item \textbf{Dissecting VoLTE: Exploiting Free Data Channels and Security Problems} \\
        {\small
            \iam{Dongkwan Kim} \\
            M.S. Thesis, KAIST
        }
        \hfill {\small Daejeon, South Korea, Feb. 2016}

    \item \textbf{Breaking and Fixing VoLTE: Exploiting Hidden Data Channels and Mis-implementations} \\
        {\small
            \iam{Dongkwan Kim$^\ast$}, Hongil Kim$^\ast$, Minhee Kwon, Hyungseok Han, Yeongjin Jang, Dongsu Han, Taesoo Kim, and Yongdae Kim \\
            Proceedings of the 22nd ACM Conference on Computer and Communications Security (CCS'15) \\
            Acceptance rate: 19.81\% (128 of 646)
        }
        \hfill {\small Denver, CO, Oct. 2015}

    \item \textbf{BurnFit: Analyzing and Exploiting Wearable Devices} \\
        {\small
            \iam{Dongkwan Kim}, Suwan Park, Kibum Choi, and Yongdae Kim \\
            Proceedings of the 16th International Workshop on Information Security Applications (WISA'15) \\
            Best Paper Award
        }
        \hfill {\small Jeju, South Korea, Aug. 2015}

    \item \textbf{Rocking Drones with Intentional Sound Noise on Gyroscopic Sensors} \\
        {\small
            Yunmok Son, Hocheol Shin, \iam{Dongkwan Kim}, Youngseok Park, Juhwan Noh, Kibum Choi, Jungwoo Choi, and Yongdae Kim \\
            Proceedings of the 24th USENIX Security Symposium (Security'15) \\
            Acceptance rate: 15.73\% (67 of 426)
        }
        \hfill {\small Austin, TX, Aug. 2015}

    \item \textbf{Analyzing Security of Korean USIM-based PKI Certificate Service} \\
        {\small
            Shinjo Park, Suwan Park, Insu Yun, \iam{Dongkwan Kim}, and Yongdae Kim \\
            Proceedings of the 15th International Workshop on Information Security Applications \\ (WISA'14)
        }
        \hfill {\small Jeju, South Korea, Aug. 2014}

    \item \textbf{High-speed Automatic Segmentation of Intravascular Stent Struts in Optical Coherence Tomography Images} \\
        {\small
            Myounghee Han, \iam{Dongkwan Kim}, Wang-Yuhl Oh, and Sukyoung Ryu \\
            Proceedings of SPIE Biomedical Optics, Photonics West 2013 (BiOS'13)
        }
        \hfill {\small San Francisco, CA, Feb. 2013}
\end{enumerate}


% \clearpage
% ========================================
% Domestic Publications
% ========================================
\sectiontitle{Publications (Domestic, South Korea)}
\begin{enumerate}[leftmargin=1.5em]
    \item \textbf{Video Service Identification Attack in LTE by Monitoring Encrypted Traffic} \\
        {\small
            Mincheol Son, Sangwook Bae, \iam{Dongkwan Kim}, Jiho Lee, CheolJun Park, BeomSeok Oh, Sooel Son, and Yongdae Kim \\
            Proceedings of Symposium of the Korean Institute of Communications and Information Sciences \\ (KCIS'21)
        }
        \hfill {\small Virtual, Jun. 2021}
    
    \item \textbf{Standard-based User Identifier Mapping Attack Prevention Method for LTE Network} \\
        {\small
            CheolJun Park, Sangwook Bae, Jiho Lee, Mincheol Son, \iam{Dongkwan Kim},
            Sooel Son, and Yongdae Kim \\
            Conference on Information Security and Cryptography Winter (CISC-W’20) \\
            Best Paper Award
        }
        \hfill {\small South Korea, Nov. 2020}

    \item \textbf{VoLTEFuzz: Framework for Comprehensive Analysis of SIP in VoLTE} \\
        {\small
            Seokbin Yun, Sangwook Bae, Mincheol Son, \iam{Dongkwan Kim}, Jiho Lee, CheolJun Park, Yeongbin Hwang, and Yongdae Kim \\
            Conference on Information Security and Cryptography Winter (CISC-W’20)
        }
        \hfill {\small South Korea, Nov. 2020}
        
    \item \textbf{Firm-Pot: Large-scale Firmware Honey-Pot for Malware Analysis} \\
        {\small
            Minguen Kim, Eunsoo Kim, \iam{Dongkwan Kim}, and Yongdae Kim \\
            Conference on Information Security and Cryptography Winter (CISC-W’18)
        }
        \hfill {\small South Korea, Dec. 2018}
        
    \item \textbf{TVT: Typed Virtual Table for Mitigating VTable Hijacking} \\
        {\small
            Jeongoh Kyea, Eunsoo Kim, \iam{Dongkwan Kim}, and Yongdae Kim \\
            Conference on Information Security and Cryptography Winter (CISC-W’17)
        }
        \hfill {\small South Korea, Dec. 2017}
        
    \item \textbf{Design and Implementation of GPS Spoofer Software} \\
        {\small
            Juhwan Noh, \iam{Dongkwan Kim}, and Yongdae Kim \\
            Conference on Information Security and Cryptography Summer (CISC-S’15)
        }
        \hfill {\small South Korea, Jun. 2015}
        
    \item \textbf{Security Analysis of USIM-based certificate service in Korea} \\
        {\small
            Shinjo Park, Suwan Park, Insu Yun, \iam{Dongkwan Kim}, and Yongdae Kim \\
            Conference on Information Security and Cryptography Summer (CISC-S’14)
        }
        \hfill {\small South Korea, Jun. 2014}
        
    \item \textbf{Security Analysis of Femtocells in Korea} \\
        {\small
            Eunsoo Kim, \iam{Dongkwan Kim}, Youjin Lee, Shinjo Park, and Yongdae Kim \\
            Conference on Information Security and Cryptography Summer (CISC-S’14)
        }
        \hfill {\small South Korea, Jun. 2014}
\end{enumerate}


% \clearpage
% ========================================
% Invited Presentations
% ========================================
\sectiontitle{Invited Talks}

\begin{timeitemize}{AI Security Primer: Red Team Perspectives on Navigating New Threats and Safeguarding AI Frontier}{}
    \timeitem{Hyundai Motors Group Security Center}{Seoul, South Korea, Jan. 2025}
    \timeitem{AI Security Lecture for the SK Telecom Information Security Team}{Seoul, South Korea, Jul. 2024}
    \timeitem{.HACK Conference by Theori}{Seoul, South Korea, May. 2024}
\end{timeitemize}

\begin{timeitemize}{Scaling up Vulnerability Analysis of IoT Devices with Heuristics and Binary Code Similarity}{}
    \timeitem{Technology Exchange Meeting between Samsung Mobile Security Team and Hyundai Motor Company Vehicle Cyber Security Team}{Seoul, South Korea, Jul. 2024}
    \timeitem{Colloquium at School of Cybersecurity, Korea University}{Seoul, South Korea, Oct. 2023}
\end{timeitemize}

\begin{timeitemize}{Peeking over Industry's Patch Gap: Case Study of Samsung SmartTV's Web Browser}{}
    \timeitem{KAIST-Samsung SDS Tech Seminar}{Daejeon, South Korea, Mar. 2023}
\end{timeitemize}

\begin{timeitemize}{BaseSpec: Comparative Analysis of Baseband Software and
        Cellular Specifications for L3 Protocols}{}
    \timeitem{Annual Network and Distributed System Security Symposium}{Virtual, Feb. 2021}
    \timeitem{KAIST-CISPA Workshop}{Seoul, South Korea, Aug. 2019}
\end{timeitemize}

\subsectiontitle{Breaking and Fixing VoLTE: Exploiting Hidden Data Channels and Mis-implementations}{} \\
\begin{timeitemize}{A.k.a. Dissecting VoLTE: Exploiting Free Data Channels and Security Problems}{}
    \timeitem{GSMA RCS/VoLTE Security Regulatory workshop}{Toronto, Canada, Sep. 2016}
    \timeitem{A3 Foresight Program Annual Workshop}{Okinawa, Japan, Feb. 2016}
    \timeitem{Chaos Communication Congress (CCC) Conference (32C3)}{Hamburg, Germany, Dec. 2015}
    \timeitem{National Security Research}{Daejeon, South Korea, Nov. 2015}
    \timeitem{Power of Community (PoC) Conference}{Seoul, South Korea, Nov. 2015}
    \timeitem{ACM Conference on Computer and Communications Security (CCS)}{Denver, CO, Oct. 2015}
    \timeitem{Seminar at the Georgia Institute of Technology}{Atlanta, GA, Oct. 2015}
\end{timeitemize}

\begin{timeitemize}{BurnFit: Analyzing and Exploiting Wearable Devices}{}
    \timeitem{16th WISA}{Jeju, South Korea, Aug. 2015}
\end{timeitemize}

\begin{timeitemize}{International CTF Challenge Solving}{}
    \timeitem{NetSec-KR}{Seoul, South Korea, Apr. 2013}
\end{timeitemize}


% ========================================
% Professional Activities
% ========================================
\sectiontitle{Professional Activities}

\begin{timeitemize}{Secondary Reviewer (Security)}{}
    \timeitem{IEEE Symposium on Security and Privacy (Oakland)}{2021}
    \timeitem{USENIX Security Symposium (Security)}{2019 -- 2021}
    \timeitem{Network and Distributed System Security Symposium (NDSS)}{2017 -- 2018, 2020 -- 2021}
    \timeitem{ACM Conference on Computer and Communications Security (CCS)}{2017, 2019 -- 2021}
    \timeitem{IEEE European Symposium on Security and Privacy (EuroS\&P)}{2016, 2018, 2020}
    \timeitem{ACM ASIA Conference on Computer and Communications Security (ASIACCS)}{2016 -- 2017, 2019 -- 2020}
    \timeitem{The WEB Conference (WWW)}{2018, 2020}
    \timeitem{International Symposium on Research in Attacks, Intrusions and Defenses (RAID)}{2017}
    \timeitem{IEEE Symposium on Privacy-Aware Computing (PAC)}{2017}
\end{timeitemize}

\begin{timeitemize}{Secondary Reviewer (System)}{}
    \timeitem{ACM Symposium on Operating Systems Principles (SOSP)}{2019}
    \timeitem{Symposium on Operating Systems Design and Implementation (OSDI)}{2016}
\end{timeitemize}


\begin{timeitemize}{External Security Consultant}{}
    \timeitem{KAIST Computer Emergency Response Team}{Sep. 2010 -- Feb. 2022}
\end{timeitemize}


% \clearpage
% ========================================
% Projects
% ========================================
\sectiontitle{Participated Projects}
($\ast$: participated as a project leader)

\begin{timeitemize}{Industrial Projects}{}
    \timeitem{An Industry-academia Task with Samsung Electronics Device Solutions Business}{Jun. 2020 -- Aug. 2020}
    \subitem{Samsung Electronics}

    \timeitem{$^\ast$Organizing 2018 Samsung Capture-the-flag (SCTF)}{Apr. 2018 -- Oct. 2018}
    \subitem{Samsung Electronics}

    \timeitem{$^\ast$Organizing 2017 Samsung Capture-the-flag (SCTF)}{Dec. 2016 -- Dec. 2017}
    \subitem{Samsung Electronics}

    \timeitem{A Study on the Security Vulnerability Analysis and Response Method of LTE \\ Networks}{Aug. 2016 -- Jul. 2017}
    \subitem{SK Telecom}

    \timeitem{A Security Vulnerability Analysis of Smartcar Core Modules}{Jul. 2016 -- Jun. 2017}
    \subitem{Hyundai NGV}

    \timeitem{A Study on the Security Analysis and Response Method of LTE Networks}{Aug. 2015 -- Apr. 2016}
    \subitem{SK Telecom}

    \timeitem{A Security Analysis of Samsung SmartTV 2014}{Feb. 2014 -- Dec. 2015}
    \subitem{Samsung Electronics}
\end{timeitemize}

\begin{timeitemize}{International Projects}{}
    \timeitem{$^\ast$Cyber Physical Analysis of System Software Survivability by Stimulating Sensors \\ on Drones}{Jun. 2020 -- Feb. 2022}
    \subitem{Air Force Office of Scientific Research (AFOSR), Air Force Research \\
    Laboratory (AFRL)}
\end{timeitemize}

\begin{timeitemize}{Governmental Projects}{}
    \timeitem{$^\ast$A Study on the Android-based Security Analysis Technology}{May 2020 -- Dec. 2020}
    \subitem{National Security Research (NSR)}

    \timeitem{A Study on the Security of Random Number Generator and Embedded Devices}{Jul. 2017 -- Jun. 2019}
    \subitem{Institute for Information \& Communications Technology Planning \& \\ Evaluation (IITP)}

    \timeitem{$^\ast$A Study on the Firmware Emulation Technology for Linux-based Routers}{May 2017 -- Oct. 2017}
    \subitem{NSR}

    \timeitem{A Development of Automated Reverse Engineering and Vulnerability Detection \\ Base Technology through Binary Code Analysis}{Apr. 2016 -- Dec. 2018}
    \subitem{IITP}

    \timeitem{$^\ast$A CAPTCHA Design based on Human Perception Characteristics}{Apr. 2016 -- Dec. 2016}
    \subitem{KAIST}

    \timeitem{$^\ast$A Study on the Vulnerability Analysis Method of Domestic/International \\ Smartcars}{Apr. 2015 -- Nov. 2015}
    \subitem{NSR}

    \timeitem{A Study on the Analysis of Technology and Security Threats in LTE Femtocell}{Sep. 2013 -- Jan. 2014}
    \subitem{Korea Internet \& Security Agency (KISA)}

    \timeitem{A Study on the Analysis and Response Method of Vulnerabilities in Network \\ Devices}{Mar. 2013 -- Dec. 2013}
    \subitem{NSR}

    \timeitem{A Study on the Vulnerability Analysis of Network Devices}{Apr. 2011 -- Oct. 2011}
    \subitem{NSR}
\end{timeitemize}



% \clearpage
% ========================================
% Other Activities
% ========================================
\sectiontitle{Other Activities}

\begin{timeitemize}{}{}
    \vspace{-1.0em}
    % \timeitem{Student Representative of School of Computing, KAIST}{Feb. 2011 -- Dec. 2013}
% \end{timeitemize}


% ========================================
% Teaching Experience
% ========================================
% \sectiontitle{Teaching Experience}

% \begin{timeitemize}{}{}
    % \vspace{-1.0em}
    \timeitem{Teaching Assistant, Introduction to Electronics Design Lab. (EE305), KAIST}{Fall 2019}
    \timeitem{Teaching Assistant, Discrete Methods for Electrical Engineering (EE213), KAIST}{Spring 2017}
    \timeitem{Teaching Assistant, Network Programming (EE324), KAIST}{Fall 2016}
    \timeitem{Teaching Assistant, Cryptography Engineering (EE817/IS893), KAIST}{Spring 2016}
    \timeitem{Teaching Assistant, Security 101: Think Like an Adversary (EE515/IS523), KAIST}{Fall 2015}
    \timeitem{Student Representative of School of Computing, KAIST}{Feb. 2011 -- Dec. 2013}
    \timeitem{Head Instructor, Information Security 101 for Freshmen (HSS062), KAIST}{Sep. 2011 -- Feb. 2013}
    \timeitem{Teaching Assistant, Information Security 101 for Freshmen (HSS062), KAIST}{Sep. 2010 -- Aug. 2011}
\end{timeitemize}



% ========================================
% List of References
% ========================================
% \clearpage
\sectiontitle{List of References}

\begin{timeitemize}{Dr. Yongdae Kim}{}
    \item Director, Cyber Security Research Center (CSRC), KAIST
    \item Professor, School of Electrical Engineering and Graduate School of Information Security, KAIST
    \item Email: \email{yongdaek@kaist.ac.kr}
    \item Homepage: \url{https://syssec.kaist.ac.kr/~yongdaek/}
\end{timeitemize}

\begin{timeitemize}{Dr. Taesoo Kim}{}
    \item Professor, School of Cybersecurity and Privacy (SCP) and Computer Science (SCS), Georgia Tech
    \item Email: \email{taesoo@gatech.edu}
    \item Homepage: \url{https://taesoo.kim/}
\end{timeitemize}

\begin{timeitemize}{Dr. Sang Kil Cha}{}
    \item Director, Cyber Security Research Center (CSRC), KAIST
    \item Associate Professor, School of Computing and Graduate School of Information Security, KAIST
    \item Email: \email{sangkilc@kaist.ac.kr}
    \item Homepage: \url{https://softsec.kaist.ac.kr/~sangkilc/}
\end{timeitemize}

\begin{timeitemize}{Dr. Sooel Son}{}
    \item Associate Professor, School of Computing and Graduate School of Information Security, KAIST
    \item Email: \email{sl.son@kaist.ac.kr}
    \item Homepage: \url{https://sites.google.com/site/ssonkaist/}
\end{timeitemize}

\begin{timeitemize}{Dr. Yeongjin Jang}{}
    \item Principal Software Engineer, Samsung Research America
    \item Email: \email{y.jang1@samsung.com}
    \item Homepage: \url{https://www.unexploitable.systems/}
\end{timeitemize}

\begin{timeitemize}{Dr. Insu Yun}{}
    \item Associate Professor, School of Electrical Engineering, KAIST
    \item Email: \email{insuyun@kaist.ac.kr}
    \item Homepage: \url{https://insuyun.github.io/}
\end{timeitemize}
